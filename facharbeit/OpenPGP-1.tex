\documentclass[12pt]{article}
\usepackage[german]{babel}
\usepackage{amsmath}

\title{Mathematische Hintergr\"unde und praktische Anwendung asymmetrischer Kryptographie am Beispiel von RSA und OpenPGP}
\input{pii.tex}

\begin{document}
\maketitle
\thispagestyle{empty}
\newpage
\tableofcontents
\newpage

\section{Einleitung}
Asymmetrische Kryptographie ist toll. Damit kann man vieles machen (verschl"usseln, unterschreiben). [Bild von Wikipedia] Im "`digitalen Zeitalter"' ist das wichtig. (TLS, IoT). OpenPGP (Phil Zimmermann) ist neben S/MIME das Standardprotokoll f"ur krypto-behandelte E-Mails, hat aber auch z.B. WoT. Ich habe einen Teil davon implementiert.

\section{Mathe hinter RSA}
Nachrichten und Schl"ussel sind Zahlen.
Hier wahrscheinlich sehr viel aus dem RSA-Paper zitieren.
\subsection{Wie Schl"ussel aussehen}
Verh"altnis zwischen "offentlichem und privaten Exponenten.
\subsubsection{Komische Funktionen}
$\phi$ / $\lambda$
\subsubsection{(vielleicht) modulare Arithmetik}
\subsection{Ver-/Entschl"usselungsprozess}
Das hier wird wahrscheinlich "Uberschneidungen mit "`Wie Schl"ussel aussehen"' haben.
\subsection{Cool! Damit kann man auch unterschreiben}
\subsection{Sicherheit}
Warum ist das Problem schwer zu l"osen?

\section{(vielleicht) Effiziente Berechnung von $a^b \mod n$}
Sonst w"are das alles viel zu langsam.

\section{(vielleicht) Der PKCS\#1-Standard}
Warum man Padding benutzt. Entweder nur die f"ur OpenPGP benutzten Teile oder alles. Wenn das sehr kurz wird, kann es auch unter OpenPGP.

\section{OpenPGP}
\subsection{"Ubersicht}
Was macht OpenPGP?
Wahrscheinlich viel aus dem OpenPGP-Standard zitieren. 
\subsection{Meine Implementation}
Was habe ich implementiert? Wie habe ich (den RSA-Teil) implementiert?
Vielleicht RSA als eigene Haupt"uberschrift. Dann vielleicht \verb|pow(a, b, n)| und PKCS\#1 nicht separat, sondern dadrunter.
Wahrscheinlich auch viel aus dem OpenPGP-Standard zitieren. 
\subsection{Beispiele}
Es funktioniert wirklich!

\section{Ausleitung}
Facharbeitshinweise:
\begin{quote}
Der Schluss sollte die Fragestellung aus der Einleitung aufgreifen und die wichtigsten Untersuchungsergebnisse zusammenfassen. Er kann den Untersuchungsgegenstand in größere Zusammenhänge einordnen und einen Ausblick auf künftige Entwicklungen enthalten. Auch kann hier das methodische Vorgehen kritisch reflektiert werden.
\end{quote}

\appendix

\section{kompletter Quellcode}

\section{Beispieldateien}

\section{weiteres Material}

\section{Literatur}

\end{document}
